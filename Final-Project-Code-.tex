\PassOptionsToPackage{unicode=true}{hyperref} % options for packages loaded elsewhere
\PassOptionsToPackage{hyphens}{url}
%
\documentclass[]{article}
\usepackage{lmodern}
\usepackage{amssymb,amsmath}
\usepackage{ifxetex,ifluatex}
\usepackage{fixltx2e} % provides \textsubscript
\ifnum 0\ifxetex 1\fi\ifluatex 1\fi=0 % if pdftex
  \usepackage[T1]{fontenc}
  \usepackage[utf8]{inputenc}
  \usepackage{textcomp} % provides euro and other symbols
\else % if luatex or xelatex
  \usepackage{unicode-math}
  \defaultfontfeatures{Ligatures=TeX,Scale=MatchLowercase}
\fi
% use upquote if available, for straight quotes in verbatim environments
\IfFileExists{upquote.sty}{\usepackage{upquote}}{}
% use microtype if available
\IfFileExists{microtype.sty}{%
\usepackage[]{microtype}
\UseMicrotypeSet[protrusion]{basicmath} % disable protrusion for tt fonts
}{}
\IfFileExists{parskip.sty}{%
\usepackage{parskip}
}{% else
\setlength{\parindent}{0pt}
\setlength{\parskip}{6pt plus 2pt minus 1pt}
}
\usepackage{hyperref}
\hypersetup{
            pdftitle={Time Series and Forecasting Final Project},
            pdfauthor={Connor O'Brien, Ross Cole, Zakir Bekenov},
            pdfborder={0 0 0},
            breaklinks=true}
\urlstyle{same}  % don't use monospace font for urls
\usepackage[margin=1in]{geometry}
\usepackage{color}
\usepackage{fancyvrb}
\newcommand{\VerbBar}{|}
\newcommand{\VERB}{\Verb[commandchars=\\\{\}]}
\DefineVerbatimEnvironment{Highlighting}{Verbatim}{commandchars=\\\{\}}
% Add ',fontsize=\small' for more characters per line
\usepackage{framed}
\definecolor{shadecolor}{RGB}{248,248,248}
\newenvironment{Shaded}{\begin{snugshade}}{\end{snugshade}}
\newcommand{\AlertTok}[1]{\textcolor[rgb]{0.94,0.16,0.16}{#1}}
\newcommand{\AnnotationTok}[1]{\textcolor[rgb]{0.56,0.35,0.01}{\textbf{\textit{#1}}}}
\newcommand{\AttributeTok}[1]{\textcolor[rgb]{0.77,0.63,0.00}{#1}}
\newcommand{\BaseNTok}[1]{\textcolor[rgb]{0.00,0.00,0.81}{#1}}
\newcommand{\BuiltInTok}[1]{#1}
\newcommand{\CharTok}[1]{\textcolor[rgb]{0.31,0.60,0.02}{#1}}
\newcommand{\CommentTok}[1]{\textcolor[rgb]{0.56,0.35,0.01}{\textit{#1}}}
\newcommand{\CommentVarTok}[1]{\textcolor[rgb]{0.56,0.35,0.01}{\textbf{\textit{#1}}}}
\newcommand{\ConstantTok}[1]{\textcolor[rgb]{0.00,0.00,0.00}{#1}}
\newcommand{\ControlFlowTok}[1]{\textcolor[rgb]{0.13,0.29,0.53}{\textbf{#1}}}
\newcommand{\DataTypeTok}[1]{\textcolor[rgb]{0.13,0.29,0.53}{#1}}
\newcommand{\DecValTok}[1]{\textcolor[rgb]{0.00,0.00,0.81}{#1}}
\newcommand{\DocumentationTok}[1]{\textcolor[rgb]{0.56,0.35,0.01}{\textbf{\textit{#1}}}}
\newcommand{\ErrorTok}[1]{\textcolor[rgb]{0.64,0.00,0.00}{\textbf{#1}}}
\newcommand{\ExtensionTok}[1]{#1}
\newcommand{\FloatTok}[1]{\textcolor[rgb]{0.00,0.00,0.81}{#1}}
\newcommand{\FunctionTok}[1]{\textcolor[rgb]{0.00,0.00,0.00}{#1}}
\newcommand{\ImportTok}[1]{#1}
\newcommand{\InformationTok}[1]{\textcolor[rgb]{0.56,0.35,0.01}{\textbf{\textit{#1}}}}
\newcommand{\KeywordTok}[1]{\textcolor[rgb]{0.13,0.29,0.53}{\textbf{#1}}}
\newcommand{\NormalTok}[1]{#1}
\newcommand{\OperatorTok}[1]{\textcolor[rgb]{0.81,0.36,0.00}{\textbf{#1}}}
\newcommand{\OtherTok}[1]{\textcolor[rgb]{0.56,0.35,0.01}{#1}}
\newcommand{\PreprocessorTok}[1]{\textcolor[rgb]{0.56,0.35,0.01}{\textit{#1}}}
\newcommand{\RegionMarkerTok}[1]{#1}
\newcommand{\SpecialCharTok}[1]{\textcolor[rgb]{0.00,0.00,0.00}{#1}}
\newcommand{\SpecialStringTok}[1]{\textcolor[rgb]{0.31,0.60,0.02}{#1}}
\newcommand{\StringTok}[1]{\textcolor[rgb]{0.31,0.60,0.02}{#1}}
\newcommand{\VariableTok}[1]{\textcolor[rgb]{0.00,0.00,0.00}{#1}}
\newcommand{\VerbatimStringTok}[1]{\textcolor[rgb]{0.31,0.60,0.02}{#1}}
\newcommand{\WarningTok}[1]{\textcolor[rgb]{0.56,0.35,0.01}{\textbf{\textit{#1}}}}
\usepackage{graphicx,grffile}
\makeatletter
\def\maxwidth{\ifdim\Gin@nat@width>\linewidth\linewidth\else\Gin@nat@width\fi}
\def\maxheight{\ifdim\Gin@nat@height>\textheight\textheight\else\Gin@nat@height\fi}
\makeatother
% Scale images if necessary, so that they will not overflow the page
% margins by default, and it is still possible to overwrite the defaults
% using explicit options in \includegraphics[width, height, ...]{}
\setkeys{Gin}{width=\maxwidth,height=\maxheight,keepaspectratio}
\setlength{\emergencystretch}{3em}  % prevent overfull lines
\providecommand{\tightlist}{%
  \setlength{\itemsep}{0pt}\setlength{\parskip}{0pt}}
\setcounter{secnumdepth}{0}
% Redefines (sub)paragraphs to behave more like sections
\ifx\paragraph\undefined\else
\let\oldparagraph\paragraph
\renewcommand{\paragraph}[1]{\oldparagraph{#1}\mbox{}}
\fi
\ifx\subparagraph\undefined\else
\let\oldsubparagraph\subparagraph
\renewcommand{\subparagraph}[1]{\oldsubparagraph{#1}\mbox{}}
\fi

% set default figure placement to htbp
\makeatletter
\def\fps@figure{htbp}
\makeatother


\title{Time Series and Forecasting Final Project}
\author{Connor O'Brien, Ross Cole, Zakir Bekenov}
\date{6/4/2021}

\begin{document}
\maketitle

\hypertarget{introduction}{%
\subsection{Introduction}\label{introduction}}

In this project, we will model monthly BLS jobs data 12 months out in
three different ways.

\hypertarget{part-one-var-model-connor}{%
\subsection{Part One: VAR Model
(Connor)}\label{part-one-var-model-connor}}

Load Pagkages

\begin{verbatim}
## Registered S3 method overwritten by 'quantmod':
##   method            from
##   as.zoo.data.frame zoo
\end{verbatim}

\begin{verbatim}
## Loading required package: MASS
\end{verbatim}

\begin{verbatim}
## Loading required package: strucchange
\end{verbatim}

\begin{verbatim}
## Loading required package: zoo
\end{verbatim}

\begin{verbatim}
## 
## Attaching package: 'zoo'
\end{verbatim}

\begin{verbatim}
## The following objects are masked from 'package:base':
## 
##     as.Date, as.Date.numeric
\end{verbatim}

\begin{verbatim}
## Loading required package: sandwich
\end{verbatim}

\begin{verbatim}
## Loading required package: urca
\end{verbatim}

\begin{verbatim}
## Loading required package: lmtest
\end{verbatim}

\begin{verbatim}
## ── Attaching packages ─────────────────────────────────────────────────────────────────────────────── tidyverse 1.3.0 ──
\end{verbatim}

\begin{verbatim}
## ✓ ggplot2 3.2.1     ✓ purrr   0.3.3
## ✓ tibble  3.1.2     ✓ dplyr   1.0.5
## ✓ tidyr   1.0.0     ✓ stringr 1.4.0
## ✓ readr   1.3.1     ✓ forcats 0.4.0
\end{verbatim}

\begin{verbatim}
## ── Conflicts ────────────────────────────────────────────────────────────────────────────────── tidyverse_conflicts() ──
## x stringr::boundary() masks strucchange::boundary()
## x dplyr::filter()     masks stats::filter()
## x dplyr::lag()        masks stats::lag()
## x dplyr::select()     masks MASS::select()
\end{verbatim}

\begin{verbatim}
## 
## Attaching package: 'lubridate'
\end{verbatim}

\begin{verbatim}
## The following object is masked from 'package:base':
## 
##     date
\end{verbatim}

Load and clean data

\begin{verbatim}
## Parsed with column specification:
## cols(
##   Month = col_character(),
##   NBER_Recession = col_double(),
##   T10Y2YM = col_double(),
##   Population = col_double(),
##   PCE = col_double(),
##   Jobs = col_double(),
##   Job_Openings = col_double(),
##   Industrial_Production = col_double(),
##   Ten_yr_Treasury = col_double(),
##   Tenyr_chg = col_double(),
##   Chi_Fed_NFCI = col_double(),
##   CPI = col_double(),
##   CPI_Inflation_m_o_m = col_double(),
##   Job_growth_m_o_m = col_double()
## )
\end{verbatim}

Total U.S. (non-farm) employment over time:

\begin{Shaded}
\begin{Highlighting}[]
\KeywordTok{plot}\NormalTok{(jobs_national, }\DataTypeTok{ylab =} \StringTok{"Jobs"}\NormalTok{, }\DataTypeTok{main =} \StringTok{"U.S. Non-Farm Employment, 1976-Today"}\NormalTok{)}
\end{Highlighting}
\end{Shaded}

\includegraphics{Final-Project-Code-_files/figure-latex/unnamed-chunk-3-1.pdf}

Model: VAR with one and two lag variables

Forecast 12 months out

\begin{Shaded}
\begin{Highlighting}[]
\NormalTok{forecast_data <-}\StringTok{ }\KeywordTok{forecast}\NormalTok{(VAR_est, }\DataTypeTok{h =} \DecValTok{12}\NormalTok{)}

\NormalTok{forecasted_jobs <-}\StringTok{ }\KeywordTok{ts}\NormalTok{(forecast_data}\OperatorTok{$}\NormalTok{forecast}\OperatorTok{$}\NormalTok{jobs_national}\OperatorTok{$}\NormalTok{mean,}
                      \DataTypeTok{start =} \KeywordTok{c}\NormalTok{(}\DecValTok{2021}\NormalTok{,}\DecValTok{4}\NormalTok{),}
                           \DataTypeTok{end =} \KeywordTok{c}\NormalTok{(}\DecValTok{2022}\NormalTok{, }\DecValTok{3}\NormalTok{),}
                           \DataTypeTok{frequency =} \DecValTok{12}\NormalTok{)}

\NormalTok{jobs_since_}\DecValTok{2010}\NormalTok{ <-}\StringTok{ }\KeywordTok{ts}\NormalTok{(jobs_national[}\DecValTok{404}\OperatorTok{:}\DecValTok{538}\NormalTok{],}
                  \DataTypeTok{start =} \KeywordTok{c}\NormalTok{(}\DecValTok{2010}\NormalTok{,}\DecValTok{1}\NormalTok{),}
                           \DataTypeTok{end =} \KeywordTok{c}\NormalTok{(}\DecValTok{2021}\NormalTok{, }\DecValTok{3}\NormalTok{),}
                           \DataTypeTok{frequency =} \DecValTok{12}\NormalTok{)}

\KeywordTok{seqplot.ts}\NormalTok{(jobs_since_}\DecValTok{2010}\NormalTok{, forecasted_jobs, }\DataTypeTok{xlab =} \StringTok{"Jobs"}\NormalTok{, }\DataTypeTok{main =} \StringTok{"Actual (Black) and Forecasted (Red) U.S. Non-Famr Jobs"}\NormalTok{)}
\end{Highlighting}
\end{Shaded}

\includegraphics{Final-Project-Code-_files/figure-latex/unnamed-chunk-5-1.pdf}

Look-back

\begin{Shaded}
\begin{Highlighting}[]
\NormalTok{VAR_data_lookback <-}\StringTok{ }\KeywordTok{window}\NormalTok{(}\KeywordTok{ts.union}\NormalTok{(jobs_national, tenyr, recession, population, Ind_Production, financial_conditions, cpi, treasury_spread), }\DataTypeTok{start =} \KeywordTok{c}\NormalTok{(}\DecValTok{1976}\NormalTok{, }\DecValTok{6}\NormalTok{), }\DataTypeTok{end =} \KeywordTok{c}\NormalTok{(}\DecValTok{2018}\NormalTok{, }\DecValTok{12}\NormalTok{))}

\NormalTok{VAR_est2 <-}\StringTok{ }\KeywordTok{VAR}\NormalTok{(}\DataTypeTok{y =}\NormalTok{VAR_data_lookback, }\DataTypeTok{p =} \DecValTok{2}\NormalTok{)}

\NormalTok{forecast_data_}\DecValTok{2}\NormalTok{ <-}\StringTok{ }\KeywordTok{forecast}\NormalTok{(VAR_est2, }\DataTypeTok{h =} \DecValTok{12}\NormalTok{)}


\NormalTok{forecasted_jobs2 <-}\StringTok{ }\KeywordTok{ts}\NormalTok{(forecast_data_}\DecValTok{2}\OperatorTok{$}\NormalTok{forecast}\OperatorTok{$}\NormalTok{jobs_national}\OperatorTok{$}\NormalTok{mean,}
                      \DataTypeTok{start =} \KeywordTok{c}\NormalTok{(}\DecValTok{2019}\NormalTok{,}\DecValTok{1}\NormalTok{),}
                           \DataTypeTok{end =} \KeywordTok{c}\NormalTok{(}\DecValTok{2019}\NormalTok{, }\DecValTok{12}\NormalTok{),}
                           \DataTypeTok{frequency =} \DecValTok{12}\NormalTok{)}

\NormalTok{recent_jobs <-}\StringTok{ }\KeywordTok{ts}\NormalTok{(jobs_national[}\DecValTok{488}\OperatorTok{:}\DecValTok{523}\NormalTok{],}
                  \DataTypeTok{start =} \KeywordTok{c}\NormalTok{(}\DecValTok{2017}\NormalTok{,}\DecValTok{1}\NormalTok{),}
                           \DataTypeTok{end =} \KeywordTok{c}\NormalTok{(}\DecValTok{2019}\NormalTok{, }\DecValTok{12}\NormalTok{),}
                           \DataTypeTok{frequency =} \DecValTok{12}\NormalTok{)}

\KeywordTok{ts.plot}\NormalTok{(forecast_data_}\DecValTok{2}\OperatorTok{$}\NormalTok{forecast}\OperatorTok{$}\NormalTok{jobs_national}\OperatorTok{$}\NormalTok{mean, recent_jobs,}\DataTypeTok{gpars =} \KeywordTok{list}\NormalTok{(}\DataTypeTok{col =} \KeywordTok{c}\NormalTok{(}\StringTok{"red"}\NormalTok{, }\StringTok{"black"}\NormalTok{)))}
\end{Highlighting}
\end{Shaded}

\includegraphics{Final-Project-Code-_files/figure-latex/unnamed-chunk-6-1.pdf}

\begin{Shaded}
\begin{Highlighting}[]
\KeywordTok{accuracy}\NormalTok{(forecasted_jobs2, recent_jobs)}
\end{Highlighting}
\end{Shaded}

\begin{verbatim}
##                 ME     RMSE      MAE       MPE      MAPE      ACF1 Theil's U
## Test set -174.4352 211.4525 181.9966 -0.115731 0.1207419 0.4884208  1.229764
\end{verbatim}

\hypertarget{part-two-ross}{%
\subsection{Part Two: (Ross )}\label{part-two-ross}}

Load Packages

Model

\hypertarget{part-three-zakir}{%
\subsection{Part Three: (Zakir)}\label{part-three-zakir}}

Load Packages

Model

\begin{Shaded}
\begin{Highlighting}[]
\CommentTok{# model}
\end{Highlighting}
\end{Shaded}

\end{document}
